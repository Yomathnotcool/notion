\documentclass{article}
\usepackage[UTF8]{ctex}
\newtheorem{theorem}{定理}
\newtheorem{example}{例}
\newtheorem{definition}{定义}
\newtheorem{lemma}{引理}
\newtheorem{proposition}{命题}
\newtheorem{corollary}{推论}
\newtheorem{proof}{证明}
\usepackage{amsmath}
\usepackage{yfonts}
\usepackage{titlesec}
\usepackage{tabularx}
\usepackage{ltxtable}
\usepackage{diagbox}
\usepackage{amstext}%公式中加入文字
\usepackage{graphicx}
\usepackage{mathtools}
\usepackage{listings}
\usepackage{float}
\usepackage{makecell}
\usepackage{colortbl}
\usepackage{xcolor}
\usepackage{array}
\usepackage{latexsym}
\usepackage{amsfonts}
\usepackage{amssymb}
\usepackage{textcomp}
\usepackage{booktabs}
\usepackage{epigraph}
\usepackage{amssymb,dsfont}
\usepackage{amsbsy}
\usepackage{enumerate}
\usepackage{flowchart}
\title{记号}
\author{sheldonshelock6 }
\date{May 2021}

\begin{document}

\maketitle

\section{Introduction}
群$(G,\circ)$是由集合$G$和二元运算"$\circ$"构成的,符合以下四个性质的数学结构:
\begin{itemize}
    \item 	封闭性:	对于所有$G$中$a$, $b$,运算$a\cirbb$的结果也在$G$中。
    \item    结合律:	对于所有$G$中的$a, b$和$c$,等式 $(a\circ b)\circ  c = a\circ (b\circ c)$成立。
    \item 单位元:	存在$G$中的一个元素$e$,使得对于所有$G$中的元素$a$,总有等式
$e\circ a = a\circ e = a$ 成立。
\item 逆元:	对于每个$G$中的$a$,存在$G$中的一个元素$b$使得总有$a\circ b = b\circ a = e$,此处$e$为单位元。
\end{itemize}
环上的模的概念是对向量空间概念的推广,这里不需要要求向量空间中的标量的代数结构是数域,条件简化标量可以是环。
因此,首先模和向量空间一样是加法交换群;在标量环元素和模元素之间定义了乘积运算,并且同样要求标量环元素和模元素的乘积是符合结合律的(在与环上定义的乘法运算一起用的时候)和分配律的。
模结构与群的表示理论有着非常密切的联系。模结构也还是交换代数和同调代数的中心概念,它非常广泛的用于代数几何和代数拓扑等研究领域中。

设 $G $是一个群, $V$ 是一个线性空间, 所谓 $G$ 在 $V$ 上的表示是指 $G$ 在 $V$ 上的一个线性作用, 也就是说对每一个 $g\in G,$ 都可以给一个 $V$ 上的线性变换 $\rho (g)$ 满足 $\rho(g_1g_2) = \rho(g_1)\rho(g_2)$
对所有的 $g_1, g_2 \in G $都成立。 换一种定义方式, $G$在 $V$ 上的表示是一个群同态 $G \rightarrow GL(V )$. 这样的群表
示我们一般记为 $(\rho, V )$, 或者简单记作 $\rho$, 或者有时也简记为 $V$ . 表示的维数就是 $V $的维数, 通常情况下记号是 $\mathbf{dim} V$ 或者 $\mathbf{dim} \rho$. 如果 $(\rho, V )$ 和 $(\tau, W)$ 是$ G$ 的表示, 他们之间的同态是一个线性映射$f : V \rightarrow W$ 满足 $f(\rho(g)v) = \tau (g)f(v)$ 对所有的 $g \in G $和 $v\in V $都成立 (也就是说线性映射和群作用交换). 两个表示同构是指存在一个同态 $f$, 使得它是线性空间的同构。

Schwartz 函数指的是当$x$值趋向于无穷大时,函数值$f(x)$趋近0的速度“足够快”的函数。Schwartz空间的一个重要性质是傅里叶变换对于这个空间是一个自同构,也就是说,Schwartz函数进行傅里叶变换之后仍然会是Schwartz函数。这个性质使得可以对$\mathcal {S}$的对偶空间中的元素,也就是对偶Schwartz函数,来定义其傅里叶变换。同时因为Schwartz函数良好的性质,所以在表示论研究中经常Schwartz空间。

分布,或者称之为Schwartz 分布或者为广义函数,是数学分析中最传统函数概念的推广。分布的概念是为了让在传统微积分中不可微分函数可以进行“微分”操作。一个函数$f$是将作用在定义域上的一点,把这个点“送到”值域中对应的点$f(x)$。而分布和函数的定义方式不同,它不再是简单的作用在一个点上。分布$g$是按照一定的规则作用在试验函数上,其中试验函数一般要求是有着紧支集的无穷可微复值函数(或者实值函数)。下面我们给出试验函数的正式定义和实值分布的正式定义:
\begin{definition}
首先约定记号$C^{k}(U)$,其中$k\in\{0,1,2,...,\infty\}$。$C^{k}(U)$为定义在$U$上所有$k$阶连续可微的实值函数组成的线性空间。$C_{c}^{k}(U)$为一个$f\in C^{k}(U)$且要求$f$有紧支集组成的集合。

从而可以定义:$C^{\infty}(U)$为试验函数空间,其中$f\in C^{\infty}_{c}(U)$为定义在$U$上的试验函数。试验函数空间也可以记为$\mathcal{D}(U)$。
\end{definition}
\begin{proposition}
若$T$是定义上试验函数空间$C^{k}(U)$上的一个线性泛函,那么$T$为一个分布当且仅当其满足以下等价条件:
\begin{enumerate}
    \item 对于每一个$U$的紧子集$K\subset U$,存在常数$C>0$且$N\in\mathbb{N}$,对于所有$f\in C^{\infty}(K)$要求:
    $$|T(f)|\leq C \mathbf{sup}\{|\partial^{\alpha}f(x):x\in U,\ |\alpha|\leq N|\}$$
    \item 对于每一个紧子集$K\in U$,存在一个常数$C_{K}>0$且$N_{K}\in\mathbf{N}$,则要求所有的函数$f\in C^{\infty}_{c}(U)$的支集包含于$k$,
    $$|T(f)|\leq C_{K} \mathbf{sup}\{|\partial^{\alpha}f(x):x\in U,\ |\alpha|\leq N_{K}|\}$$
    \item 对于任意一个紧子集$K\in U$,和任意一个定义在$C^{\infty}(K)$上的函数序列$\{f_{i}^{\infty}\}_{i=1}$,若对于所有多元数组$\alpha$,$\{\partial^{\alpha}f_{i}\}^{\infty}_{i=1}$在$K$上一致收敛于0,则$T(f_{i})\rightarrow 0$。
\end{enumerate}
\end{proposition}
通过这三条性质可以判断一个线性泛函是不是一个分布,但是这仍然是非常局限的。在真正的实际应用过程,我们往往需要在试验函数空间$C^{\infty}_{c}(U)$或者$\mathcal{D}(U)$上定义拓扑,而在这样的空间$C^{\infty}_{c}(U)$或者$\mathcal{D}(U)$上定义典范拓扑,需要进一步定义凸拓扑线性空间等概念,在此不再一一解释。

代数簇,亦作代数多样体,是代数几何学上多项式集合的公共零点解的集合。代数簇是代数几何的中心研究对象。
历史上,代数基本定理建立了代数和几何之间的一个联系,它表明在复数域上的单变量的多项式由它的根的集合决定,而根集合是内在的几何对象。在此基础上,Hilbert's Nullstellensatz提供了多项式环的理想和仿射空间子集的对应关系。通过Hilbert's Nullstellensatz和相关结果,我们便能够用代数的技术手段来说明簇的几何概念,也能够用几何来更直观地展示环论中的问题。


\end{document}
